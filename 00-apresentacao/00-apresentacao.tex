%%%%%%%%%%%%%%%%%%%%%%%%%%%%%%%%%%%%%%%%%
% Beamer Presentation
% LaTeX Template
% Version 2.0 (March 8, 2022)
%
% This template originates from:
% https://www.LaTeXTemplates.com
%
% Author:
% Vel (vel@latextemplates.com)
%
% License:
% CC BY-NC-SA 4.0 (https://creativecommons.org/licenses/by-nc-sa/4.0/)
%
%%%%%%%%%%%%%%%%%%%%%%%%%%%%%%%%%%%%%%%%%

%----------------------------------------------------------------------------------------
%	PACKAGES AND OTHER DOCUMENT CONFIGURATIONS
%----------------------------------------------------------------------------------------

\documentclass[
	10pt, % Set the default font size, options include: 8pt, 9pt, 10pt, 11pt, 12pt, 14pt, 17pt, 20pt
	t, % Uncomment to vertically align all slide content to the top of the slide, rather than the default centered
	%aspectratio=169, % Uncomment to set the aspect ratio to a 16:9 ratio which matches the aspect ratio of 1080p and 4K screens and projectors
]{beamer}

\graphicspath{{Images/}{./}} % Specifies where to look for included images (trailing slash required)

\usepackage{booktabs} % Allows the use of \toprule, \midrule and \bottomrule for better rules in tables
\usepackage{graphicx}
\usepackage{caption}
\usepackage{subcaption}
\usepackage{hyperref}
\usepackage[english,brazil]{babel}
\RequirePackage[backend=biber,
style=ieee,
%style=authoryear,
%style=authoryear-comp,
%style=authoryear-ibid,
%style=authoryear-icomp,
%style=authoryear-icomp,
citestyle=authoryear,
]{biblatex}

%----------------------------------------------------------------------------------------
%	SELECT LAYOUT THEME
%----------------------------------------------------------------------------------------

% Beamer comes with a number of default layout themes which change the colors and layouts of slides. Below is a list of all themes available, uncomment each in turn to see what they look like.

%\usetheme{default}
%\usetheme{AnnArbor}
%\usetheme{Antibes}
%\usetheme{Bergen}
%\usetheme{Berkeley}
%\usetheme{Berlin}
%\usetheme{Boadilla}
%\usetheme{CambridgeUS}
%\usetheme{Copenhagen}
%\usetheme{Darmstadt}
%\usetheme{Dresden}
%\usetheme{Frankfurt}
%\usetheme{Goettingen}
%\usetheme{Hannover}
%\usetheme{Ilmenau}
%\usetheme{JuanLesPins}
%\usetheme{Luebeck}
\usetheme{Madrid}
%\usetheme{Malmoe}
%\usetheme{Marburg}
%\usetheme{Montpellier}
%\usetheme{PaloAlto}
%\usetheme{Pittsburgh}
%\usetheme{Rochester}
%\usetheme{Singapore}
%\usetheme{Szeged}
%\usetheme{Warsaw}

%----------------------------------------------------------------------------------------
%	SELECT COLOR THEME
%----------------------------------------------------------------------------------------

% Beamer comes with a number of color themes that can be applied to any layout theme to change its colors. Uncomment each of these in turn to see how they change the colors of your selected layout theme.

%\usecolortheme{albatross}
%\usecolortheme{beaver}
%\usecolortheme{beetle}
% \usecolortheme{crane}
%\usecolortheme{dolphin}
%\usecolortheme{dove}
%\usecolortheme{fly}
%\usecolortheme{lily}
%\usecolortheme{monarca}
%\usecolortheme{seagull}
%\usecolortheme{seahorse}
%\usecolortheme{spruce}
%\usecolortheme{whale}
%\usecolortheme{wolverine}

%----------------------------------------------------------------------------------------
%	SELECT FONT THEME & FONTS
%----------------------------------------------------------------------------------------

% Beamer comes with several font themes to easily change the fonts used in various parts of the presentation. Review the comments beside each one to decide if you would like to use it. Note that additional options can be specified for several of these font themes, consult the beamer documentation for more information.

\usefonttheme{default} % Typeset using the default sans serif font
%\usefonttheme{serif} % Typeset using the default serif font (make sure a sans font isn't being set as the default font if you use this option!)
%\usefonttheme{structurebold} % Typeset important structure text (titles, headlines, footlines, sidebar, etc) in bold
%\usefonttheme{structureitalicserif} % Typeset important structure text (titles, headlines, footlines, sidebar, etc) in italic serif
%\usefonttheme{structuresmallcapsserif} % Typeset important structure text (titles, headlines, footlines, sidebar, etc) in small caps serif

%------------------------------------------------

%\usepackage{mathptmx} % Use the Times font for serif text
\usepackage{palatino} % Use the Palatino font for serif text

%\usepackage{helvet} % Use the Helvetica font for sans serif text
% \usepackage[default]{opensans} % Use the Open Sans font for sans serif text
%\usepackage[default]{FiraSans} % Use the Fira Sans font for sans serif text
\usepackage[default]{lato} % Use the Lato font for sans serif text

%----------------------------------------------------------------------------------------
%	SELECT INNER THEME
%----------------------------------------------------------------------------------------

% Inner themes change the styling of internal slide elements, for example: bullet points, blocks, bibliography entries, title pages, theorems, etc. Uncomment each theme in turn to see what changes it makes to your presentation.

%\useinnertheme{default}
% \useinnertheme{circles}
\useinnertheme{rectangles}
%\useinnertheme{rounded}
%\useinnertheme{inmargin}

%----------------------------------------------------------------------------------------
%	SELECT OUTER THEME
%----------------------------------------------------------------------------------------

% Outer themes change the overall layout of slides, such as: header and footer lines, sidebars and slide titles. Uncomment each theme in turn to see what changes it makes to your presentation.

%\useoutertheme{default}
%\useoutertheme{infolines}
%\useoutertheme{miniframes}
%\useoutertheme{smoothbars}
%\useoutertheme{sidebar}
%\useoutertheme{split}
%\useoutertheme{shadow}
%\useoutertheme{tree}
%\useoutertheme{smoothtree}

%\setbeamertemplate{footline} % Uncomment this line to remove the footer line in all slides
%\setbeamertemplate{footline}[page number] % Uncomment this line to replace the footer line in all slides with a simple slide count

%\setbeamertemplate{navigation symbols}{} % Uncomment this line to remove the navigation symbols from the bottom of all slides

% \bibliography{references} % Specifies the bibliography file to include publications
% \bibliographystyle{apalike} % Specifies the bibliography style
\addbibresource{references.bib}

%----------------------------------------------------------------------------------------
%	PRESENTATION INFORMATION
%----------------------------------------------------------------------------------------

\title[DesWebII]{Desenvolvimento Web II} % The short title in the optional parameter appears at the bottom of every slide, the full title in the main parameter is only on the title page
\subtitle{Aula 00 - Apresentação da Disciplina} % Presentation subtitle, remove this command if a subtitle isn't required
\author[Fabricio Bizotto]{Prof. Fabricio Bizotto} % Presenter name(s), the optional parameter can contain a shortened version to appear on the bottom of every slide, while the main parameter will appear on the title slide
\institute[IFC]{Instituto Federal Catarinense \\ \smallskip \textit{fabricio.bizotto@ifc.edu.br}} % Your institution, the optional parameter can be used for the institution shorthand and will appear on the bottom of every slide after author names, while the required parameter is used on the title slide and can include your email address or additional information on separate lines
\date[\today]{Ciência da Computação \\ \today} % Presentation date or conference/meeting name, the optional parameter can contain a shortened version to appear on the bottom of every slide, while the required parameter value is output to the title slide

%----------------------------------------------------------------------------------------
\begin{document}

%----------------------------------------------------------------------------------------
%	TITLE SLIDE
%----------------------------------------------------------------------------------------

\begin{frame}
	\titlepage % Output the title slide, automatically created using the text entered in the PRESENTATION INFORMATION block above
\end{frame}

%----------------------------------------------------------------------------------------
%	TABLE OF CONTENTS SLIDE
%----------------------------------------------------------------------------------------
% **Unidade 1: Introdução à Arquitetura de Sistemas Web (4 horas)**\
% 1.1. Conceitos básicos de arquitetura de sistemas web\
% 1.2. Modelos arquiteturais: monolítico, cliente-servidor, arquitetura em camadas, arquitetura orientada a serviços (SOA), arquitetura orientada a microsserviços\
% 1.3. Principais desafios e tendências\
% The table of contents outputs the sections and subsections that appear in your presentation, specified with the standard \section and \subsection commands. You may either display all sections and subsections on one slide with \tableofcontents, or display each section at a time on subsequent slides with \tableofcontents[pausesections]. The latter is useful if you want to step through each section and mention what you will discuss.

\begin{frame}
	\frametitle{Roteiro} % Slide title, remove this command for no title
	
	\tableofcontents % Output the table of contents (all sections on one slide)
	%\tableofcontents[pausesections] % Output the table of contents (break sections up across separate slides)
\end{frame}

%----------------------------------------------------------------------------------------
%	PRESENTATION BODY SLIDES
%----------------------------------------------------------------------------------------

\section{Introdução a Arquitetura de Sistemas Web} % Sections are added in order to organize your presentation into discrete blocks, all sections and subsections are automatically output to the table of contents as an overview of the talk but NOT output in the presentation as separate slides

%------------------------------------------------

\subsection{Ementa e Objetivos}

\begin{frame}
	\frametitle{Ementa e Objetivos}

	\begin{block}{Ementa}
		Segurança e arquitetura de sistemas Web. Serviços Web. Integração de sistemas. Tecnologias emergentes de sistemas Web. 
	\end{block}

	\begin{block}{Objetivos}
		\begin{itemize}
			\item Capacitar os estudantes a compreender e aplicar conceitos fundamentais relacionados à arquitetura de sistemas web.
			\item Compreender os conceitos básicos de sistemas Web e segurança de serviços web.
			\item Promover o estudo sobre integração de sistemas Web.
			\item Averiguar os desafios e tendências de sistemas Web.
			\item Estudar tecnologias emergentes de sistemas Web.
			\item Estimular a aplicação dos conceitos estudados em atividades práticas.
		\end{itemize}
	\end{block}
	
\end{frame}

%------------------------------------------------

\subsection{Metodologia}

\begin{frame}
	\frametitle{Metodologia}

	\begin{itemize}
		\item Aulas expositivas dialogadas.
		\item Lista de Exercícios.
		\item Trabalhos individuais e em grupo.
		\item Desenvolvimento de projetos práticos.
		\item Seminários.
	\end{itemize}
	
\end{frame}

\subsection{Avaliação}

\begin{frame}
	\frametitle{Avaliação}

	\begin{itemize}
		\item Listas de Exercícios (20\%).
		\item Prova (30\%).
		\item Projeto Prático (50\%).
	\end{itemize}

	\begin{block}{Critérios de Avaliação}
		\begin{itemize}
			\item Cumprimento dos prazos.
			\item Qualidade dos trabalhos.
		\end{itemize}
	\end{block}

\end{frame}

\subsection{Bibliografia}

\begin{frame}
	\frametitle{Bibliografia}

	\begin{block}{Bibliografia Básica}
		\begin{itemize}
			\item {\small [1] HOGAN, Brian P. HTML 5 and CSS3: desenvolva hoje com o padrão de amanhã . Rio de Janeiro: Editora Ciência Moderna, 2012.}
			\item {\small [2] LUCKOW, Décio Heinzelmann; MELO, Alexandre Altair de. Programação Java para a Web. São Paulo, SP: Novatec, 2010.}
			\item {\small [3] NIEDERAUER, Juliano. Desenvolvendo websites com PHP: aprenda a criar websites dinâmicos e interativos com PHP e bancos de dados. 2. ed. rev. e atual. São Paulo: Novatec, 2011.}
		\end{itemize}
	\end{block}

	\begin{block}{Bibliografia Complementar}
		\begin{itemize}
			\item {\small [1] CAELUM. Desenvolvimento Web com HTML, CSS e JavaScript. Disponível em <https://www.caelum.com.br/download/caelum-html-css-javascript.pdf>. Acesso em: 03 jul. 2018.}
			\item {\small [2] FERREIRA, Elcio; EIS, Diego. HTML 5: Curso W3C Escritório Brasil. Disponível em: <http://www.w3c.br/pub/Cursos/CursoHTML5/html5-web pdf>. Acesso em: 03 jul. 2018.}
			\item {\small [3] HAVERBEKE, Marijn. JavaScript Eloquente: Uma moderna introdução ao JavaScript, programação e maravilhas digitais. 2 ed. Disponível em: <https://github.com/braziljs/eloquente-javascript>. Acesso em: 03 jul. 2018.}
			\item {\small [4] NIELSEN, Jakob; LORANGER, Hoa. Usabilidade na web: projetando websites com qualidade. Rio de Janeiro: Elsevier, Campus, 2007.}
			\item {\small [5] WATRALL, Ethan; SIARTO, Jeff. Use a cabeça! web design . Rio de Janeiro: Alta Books, 2012.}
		\end{itemize}
		
	\end{block}

\end{frame}
%------------------------------------------------

% \begin{frame}
% 	\frametitle{References}
% 	\printbibliography
% \end{frame}


%----------------------------------------------------------------------------------------

\end{document} 