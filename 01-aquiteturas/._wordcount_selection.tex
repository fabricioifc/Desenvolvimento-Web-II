- __`Acoplamento fraco`__: as camadas são independentes, o que significa que uma alteração em uma camada não afeta outras camadas. Por exemplo, uma alteração na camada de apresentação não afeta a camada de negócios.
- __`Modularidade`__: as camadas são independentes, o que significa que uma camada pode ser substituída por outra camada. Por exemplo, a camada de apresentação pode ser substituída por uma camada de serviços web.
- __`Manutenabilidade`__: a separação de responsabilidades e a modularidade facilitam a manutenção do sistema. As alterações podem ser feitas em uma camada sem afetar outras partes do sistema. 
- __`Escalabilidade`__: pode-se adicionar ou remover camadas conforme necessário para atender aos requisitos específicos do sistema. Por exemplo, pode-se adicionar uma camada de cache para melhorar o desempenho do sistema.
- __`Tecnologias`__: as camadas são independentes, o que significa que cada camada pode ser desenvolvida usando uma linguagem de programação e tecnologias diferentes. Por exemplo, a camada de apresentação pode ser desenvolvida usando HTML, CSS e JavaScript, enquanto a camada de negócios pode ser desenvolvida usando Java e a camada de dados pode ser desenvolvida usando SQL.
- __`Padrões de Comunicação`__: O acesso entre camadas geralmente segue padrões bem definidos, como chamadas de método, troca de mensagens ou serviços web, dependendo do contexto.